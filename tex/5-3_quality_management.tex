\section{Gestión de calidad}
\subsection{Planificación de la gestión de la calidad}
La calidad del producto se definirá mediante dos grupos: la calidad
mínima y la calidad extra. Los elementos de calidad mínima serán
aquellos que son indispensables para poder realizar un cierre
satisfactorio y exitoso del proyecto. Por contra, los elementos que se
agrupen dentro de la calidad extra, serán aquellos a los que se les
dedicarán recursos siempre y cuando la calidad mínima en el producto
haya sido conseguida.

En el caso de que se propongan nuevas ideas, características o
requerimientos para el producto final, se tendrá que hacer una
evaluación de dichas características para decidir en qué grupo habrá de
clasificarla.

Por último, si uno de los elementos considerados de calidad mínima no
puede ser completado o desarrollado, se estudiará la posibilidad de
moverlo al grupo de elementos de calidad extra, siempre teniendo en
cuenta la opinión de los interesados del proyecto.

\subsection{Calidad mínima}
\label{sec:min:qual}
Los elementos que se incluirán en el producto, y que se consideran que
son indispensables para el mismo, son los elementos del 1 al 10, especificados 
en la sección de requerimientos \ref{enum:req:lis}.

Además, se hará hincapié en las pruebas del producto o \textit testing \textit,
con el objetivo de que el producto pueda ser considerado estable y robusto, y
que sobre todo asista en el desarrollo a la hora de modificar el susodicho
producto, indicando qué partes funcionan y cuales no a la hora de introducir
los cambios.

\subsection{Calidad añadida}
\label{sec:add:qual}
Los elementos de calidad añadida para el producto son los elementos 11,
12 y 13 especificados en la sección de requerimientos \ref{enum:req:lis}.

\subsection{Seguimiento y control de la calidad}
El seguimiento y control de la calidad se irá haciendo en conjunción con
el seguimiento y control de los requerimientos \ref{subsec:syc:scope},
el alcance \ref{subsec:syc:scope} y las actividades
\ref{subsec:syc:timeManagement}, y dependiendo del estado general del
conjunto del proyecto se tomarán las decisiones oportunas.
