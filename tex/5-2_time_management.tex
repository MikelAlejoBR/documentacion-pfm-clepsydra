\section{Gestión del tiempo}
\subsection{Planificación de la gestión del tiempo}
Una vez se hayan conocido todos los requerimientos y se haya definido el
alcance del proyecto, se procederá a realizar la primera estimación general
del mismo. Para la parte de la estimación del desarrollo del producto, se
pedirá opinión y quizás alguna directriz a algunos profesionales de la empresa,
los cuales ya han trabajado en productos similares. De esta forma se podrá
realizar una estimación más precisa. Se definirán las actividades con la
máxima precisión posible, intentando minimizar el tener que definir más tareas
a lo largo del proyecto.

En el caso en el que sea necesario modificar la lista de actividades, se
añadirán, modificarán o quitarán siempre teniendo en cuenta el alcance, los
requerimientos y los recursos disponibles para el proyecto, de forma que
no se ponga en peligro la viabilidad del mismo.


\subsection{Definición de actividades}

\subsection{Secuencia de actividades}

\subsection{Estimación de la duración de las actividades}

\subsection{Seguimiento y control}
La unidad de tiempo mínima que se utilizará para realizar el seguimiento y
control de cada actividad será de 15 minutos. Dichas actividades y el tiempo
invertido en cada una de ellas será recogido en una tabla donde quede reflejada
toda la información pertinente.
