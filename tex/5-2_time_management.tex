\section{Gestión del tiempo}
\subsection{Planificación de la gestión del tiempo}
Una vez se hayan conocido todos los requerimientos y se haya definido el
alcance del proyecto, se procederá a realizar la primera estimación general
del mismo. Para la parte de la estimación del desarrollo del producto, se
pedirá opinión y quizás alguna directriz a algunos profesionales de la empresa,
los cuales ya han trabajado en productos similares. De esta forma se podrá
realizar una estimación más precisa. Se definirán las actividades con la
máxima precisión posible, intentando minimizar el tener que definir más tareas
a lo largo del proyecto.

En el caso en el que sea necesario modificar la lista de actividades, se
añadirán, modificarán o quitarán siempre teniendo en cuenta el alcance, los
requerimientos y los recursos disponibles para el proyecto, de forma que
no se ponga en peligro la viabilidad del mismo.

\subsection{Definición de las actividades}
\begin{enumerate}

 \scriptsize
  \item Crear estructura del documento de \LaTeX .
  \item Planificación.

  \begin{enumerate}

    \item Alcance

    \begin{enumerate}

      \item Documentar la gestión del alcance.
      \item Documentar la gestión de los requerimientos.
      \item Documentar la gestión de la validación del alcance.
      \item Escribir un borrador con las preguntas a realizar en las
        entrevistas.
      \item Realizar entrevistas.

      \begin{enumerate}

        \item Realizar entrevista con los trabajadores de desarrollo.
        \item Realizar entrevista con los trabajadores de gestión.
        \item Realizar entrevista con los trabajadores de contabilidad.

      \end{enumerate}

      \item Procesar respuestas y elaborar los requerimientos.
      \item Generar un alcance inicial.
      \item Convocar una reunión con los interesados para validar el alcance.
      \item Corregir los requerimientos y el alcance.
      \item Documentar la gestión del seguimiento y control del alcance.
      \item Generar un \gls{edt}.

    \end{enumerate}

    \item Tiempo

    \begin{enumerate}

      \item Documentar la gestión del tiempo del proyecto.
      \item Realizar una definición de actividades.
      \item Analizar y elaborar una secuencia de actividades.
      \item Estimar la duración de las actividades.
      \item Identificar y recoger los hitos.
      \item Redactar la gestión del seguimiento y control de las actividades.

    \end{enumerate}

    \item Calidad

    \begin{enumerate}

      \item Documentar la gestión de la calidad del proyecto.
      \item Especificar qué es considerada como calidad mínima y qué como
        calidad añadida.
      \item Redactar la gestión del seguimiento y control de la calidad.

    \end{enumerate}

    \item Riesgos

    \begin{enumerate}

      \item Documentar la gestión de los riesgos del \gls{pfm}.
      \item Analizar y elaborar una lista de posibles riesgos.
      \item Elaborar planes de contingencia.

    \end{enumerate}

  \end{enumerate}

  \item Gestión
  \label{item:gestion}

  \begin{enumerate}

    \item Realizar seguimiento y control del alcance.
    \item Realizar seguimiento y control del tiempo.
    \item Realizar seguimiento y control de la calidad.
    \item Realizar seguimiento y control de los riesgos.

  \end{enumerate}

  \item Diseño
  \label{item:diseno}

  \begin{enumerate}

    \item Realizar un \gls{modentrel} del dominio.
    \item Diseñar y documentar la \gls{api}. \label{item:disdocapi}

  \end{enumerate}

  \item Desarrollo

  \begin{enumerate}

    \item Formación

    \begin{enumerate}

      \item Obtener información de los aspectos básicos de las \gls{api}.
      \item Aprender los fundamentos del \gls{tdd}.
      \item Aprender a usar el \gls{webframework} \textit{Symfony}
        \footnote{\url{https://symfony.com/}} con \gls{api}s.
      \item Recavar información sobre las diferentes formas de autenticación
        con una \gls{api} \gls{rest}.

    \end{enumerate}
 
    \item Implementación

    \begin{enumerate}

      \item Preparar el entorno para la instalación del \gls{webframework}.
      \item Instalar el \gls{webframework} y probar que funciona.
      \item Instalar los módulos necesarios para proveer al \gls{webframework}
        de las herramientas necesarias para desarrollar la \gls{api}.
      \item Comenzar con el desarrollo del producto teniendo en cuenta el
        \gls{modentrel}, mediante la metodología \gls{tdd}.

    \end{enumerate}


  \end{enumerate}

\end{enumerate}

\subsection{Secuencia de actividades}
Las dependencias entre actividades vienen marcadas por el orden en el que se
han definido en la sección anterior, con la excepción de la parte de
\ref{item:gestion} Gestión, la cual se puede ir realizando según vaya avanzando
el proyecto.

Lo mismo ocurre con el apartado de \ref{item:diseno} Diseño, el cual a pesar
de ser realizado antes de comenzar con la implementación del producto, es
probable que se desarrolle junto con el producto, sobre todo la actividad de
\ref{item:disdocapi} Diseñar y documentar la \gls{api}.

\subsection{Estimación de la duración de las actividades}

\begin{longtable}{|c|c|c|}
\hline
\textbf{Actividad} & \textbf{Estimación en minutos} & \textbf{Estimación en horas} \\ \hline
1 & 720 & 12 \\ \hline
2 & 1095 & 18,25 \\ \hline
2.1. & 615 & 10,25 \\ \hline
2.1.1. & 30 & 0,5 \\ \hline
2.1.2. & 30 & 0,5 \\ \hline
2.1.3. & 30 & 0,5 \\ \hline
2.1.4. & 30 & 0,5 \\ \hline
2.1.5. & 270 & 4,5 \\ \hline
2.1.5.2. & 90 & 1,5 \\ \hline
2.1.5.2. & 90 & 1,5 \\ \hline
2.1.5.3. & 90 & 1,5 \\ \hline
2.1.6. & 60 & 1 \\ \hline
2.1.7. & 30 & 0,5 \\ \hline
2.1.8. & 30 & 0,5 \\ \hline
2.1.9. & 30 & 0,5 \\ \hline
2.1.10 & 30 & 0,5 \\ \hline
2.1.11. & 45 & 0,75 \\ \hline
2.2. & 270 & 4,5 \\ \hline
2.2.1. & 30 & 0,5 \\ \hline
2.2.2. & 60 & 1 \\ \hline
2.2.3. & 30 & 0,5 \\ \hline
2.2.4. & 60 & 1 \\ \hline
2.2.5. & 30 & 0,5 \\ \hline
2.2.6. & 60 & 1 \\ \hline
2.3. & 105 & 1,75 \\ \hline
2.3.1. & 30 & 0,5 \\ \hline
2.3.2. & 45 & 0,75 \\ \hline
2.3.3. & 30 & 0,5 \\ \hline
2.4. & 105 & 1,75 \\ \hline
2.4.1. & 30 & 0,5 \\ \hline
2.4.2. & 45 & 0,75 \\ \hline
2.4.3. & 30 & 0,5 \\ \hline
3. & 720 & 12 \\ \hline
3.1. & 180 & 3 \\ \hline
3.2. & 180 & 3 \\ \hline
3.3. & 180 & 3 \\ \hline
3.4. & 180 & 3 \\ \hline
4. & 300 & 5 \\ \hline
4.1. & 60 & 1 \\ \hline
4.2. & 240 & 4 \\ \hline
5. & 22020 & 367 \\ \hline
5.1. & 720 & 12 \\ \hline
5.1.1. & 120 & 2 \\ \hline
5.1.2. & 120 & 2 \\ \hline
5.1.3. & 300 & 5 \\ \hline
5.1.4. & 180 & 3 \\ \hline
5.2. & 21300 & 355 \\ \hline
5.2.1. & 60 & 1 \\ \hline
5.2.2. & 180 & 3 \\ \hline
5.2.3. & 60 & 1 \\ \hline
5.2.4. & 21000 & 350 \\ \hline
\multicolumn{2}{|c|}{Horas totales} & 414,25 \\ \hline
\end{longtable}

\subsection{Hitos}
\begin{itemize}
  \item Inicio del proyecto: 2017-11-13
  \item Inicio de la fase de desarrollo: 2017-12-01
  \item El segundo integrante se incorpora: 2018-01-15
  \item Fin de la fase de desarrollo: 2017-03-28
  \item Fin del proyecto: 2017-04-16
\end{itemize}

\subsection{Seguimiento y control}
\label{subsec:syc:timeManagement}
La unidad de tiempo mínima que se utilizará para realizar el seguimiento y
control de cada actividad será de 15 minutos. Dichas actividades y el tiempo
invertido en cada una de ellas será recogido en una tabla donde quede reflejada
toda la información pertinente.
