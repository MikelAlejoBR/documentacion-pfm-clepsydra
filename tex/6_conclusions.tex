\chapter{Cierre de proyecto y conclusiones}
\section{Seguimiento y control}
\subsection{Alcance}
El alcance especificado ha sido cumplido minuciosamente, y la decisión de haber
separado el proyecto en dos más pequeños parece haber sido un acierto, ya que
me ha permitido desarrollar la aplicación junto con todas sus pruebas
correspondientes, dando como resultado un \glslink{backend}{backend} que podría
ser considerado como estable y sólido.

\subsection{Tiempo}
\label{sec:syc:time}
El proyecto necesitó un total de 484 horas y 25 minutos para completarse, que
se desglosan en 59 horas y 25 minutos de gestión y desarrollo de la
documentación, 385 horas de desarrollo, y 40 horas más para la finalización de
la documentación, retoques finales y preparación de la defensa. La siguiente
lista lo resume:

\begin{itemize}
    \item Gestión: 59 horas y 25 minutos.
    \begin{itemize}
        \item Noviembre: 53 horas y 55 minutos.
        \item Diciembre: 20 horas y 15 minutos.
        \item Marzo: 4 horas y 15 minutos.
    \end{itemize}
    \item Desarrollo: 385 horas.
    \begin{itemize}
        \item Diciembre: 20 horas y 45 minutos.
        \item Enero: 155 horas.
        \item Febrero: 95 horas y 30 minutos.
        \item Marzo: 113 horas y 45 minutos.
    \end{itemize}
    \item Gestión final: 40 horas.
\end{itemize}

Esto supera por 68 horas y 10 minutos estimación inicial de 416 horas y 15 
minutos la planificación de tiempo realizada para el proyecto, y por 34 horas y
25 minutos los recursos totales evaluables del proyecto: 450 horas.

\subsubsection{Fase inicial}
\textit{Nota: se puede encontrar el informe completo con las tareas detalladas
en el anexo \ref{app:reports:start}}

Los principales desvíos se han dado en la gestión al inicio del
proyecto. Si bien se estimaba que el inicio del desarrollo comenzaría hacia el
28 de Noviembre de 2017, no es hasta el 14 de Diciembre de 2017 que se tiene
constancia de la primera hora invertida en el desarrollo del producto. Esto se
traduce en 54 horas y 25 minutos, que se invirtieron en las tareas descritas en
la lista de abajo. También cabe mencionar que durante ese
periodo, la dedicación no fue exclusivamente al proyecto, sino que se compaginó
con otras tareas independientes al mismo, que fueron realizadas para la
empresa. Volviendo a la lista, el plasmar toda la
información en el documento \LaTeX llevó más tiempo del deseado, ya que tuve
que volver a aprender a utilizar \LaTeX correctamente, los problemas de
compilación, y sobre todo los problemas de corrección y adecuación de formato.

Al comienzo del proyecto me dediqué sobre todo a realizar las entrevistas a los
diferentes grupos de interesados de la aplicación, procesar las respuestas que me
dieron y plasmar los requerimientos extraídos en la sección de requerimientos \textemdash véase sección \ref{subsec:rdr} \textemdash
de este documento.

Se realizaron varias reuniones a lo largo de la fase inicial. En ellas se definió
la idea que tenían los interesados sobre lo que se quería realizar en este proyecto,
y sobre todo se usaron para validar lo que yo había recogido como requerimientos
y la definición del alcance.

Por último se dedicó tiempo a diseñar y modelar el \glslink{modentrel}{modelo entidad-relación}.

\subsubsection{Fase de desarrollo}
\paragraph{Diciembre}
\textit{Nota: se puede encontrar el informe completo con las tareas detalladas
en el anexo \ref{app:reports:december}}

En diciembre, sobre todo me dediqué a las tareas de instalación y configuración
del \glslink{webframework}{framework} Symfony, aprender sobre algunos componentes clave
y poner en práctica lo aprendido en proyectos de prueba.

\paragraph{Enero}
\label{sec:tnc:jan}
\textit{Nota: se puede encontrar el informe completo con las tareas detalladas
en el anexo \ref{app:reports:january}}

Uno de mis objetivos con la aplicación era que tuviera una documentación de la
\gls{api} clara y concisa. Es por ello que utilicé la librería
\textit{nelmio/api-doc-bundle}, la cual facilita la documentación de los
diferentes \glslink{endpoint}{endpoints} que conforman una \gls{api}.

En un principio, comencé a usar \textit{friendsofsymfony/rest-bundle} para
generar los \glslink{endpoint}{endpoints} de la \gls{api}, en vez de utilizar
API Platform. Esto se debe a que en un principio consideré más apropiado utilizar
una librería que no trajera nada por defecto, de forma que me sirviera de formación
para cuando usase el \glslink{webframework}{framework} de API Platform.

Después de esto, me dediqué a configurar el serializador \textemdash véase
\ref{sec:serializer} \textemdash \textit{jms/serializer}, básicamente porque
lo recomendaba la librería \\ \textit{friendsofsymfony/rest-bundle}.

Una vez configurado, creé el dominio en base al \glslink{modentrel}{modelo 
entidad-relación} que había diseña, ya que hasta entonces había configurado
el serializador con unas entidades simples, que me permitieran centrarme en
entender cómo trabajar con él sin tener que lidiar con la incertidumbre de
no saber si algo no funciona por causa del serializador o las entidades.

A continuación tocó crear lo que se denominan \textit{fixtures}, o datos de
prueba, con las entidades creadas, de forma que pudiera comprobar que no
existían fallos de diseño en el \glslink{modentrel}{modelo entidad-relación}.
Por desgracia sí que existían problemas no con el diseño, sino con la
implementación del mismo, así que tuve que lidiar con ellos y retocar las
entidades para que funcionaran bien con el \glslink{webframework}{framework}
Symfony.

Con las bases ya controladas, procedí a crear unos \glslink{endpoint}{endpoints}
de prueba con una de las entidades, realicé las pruebas correspondientes y contrasté
los resultados obtenidos. Obviamente hubo mucha prueba y error, pero una vez
conseguido, continué configurando el serializador para intentar limitar la
información que se devolvía dependiendo de los roles establecidos.

Debido a que la fecha de incorporación del alumno que se iba a encargar del \glslink{frontend}{frontend}
de la aplicación estaba muy cerca, me puse manos a la obra con API Platform. La
gran ventaja de este \glslink{webframework}{framework} para \gls{api}s es que
genera automáticamente todos los \glslink{endpoint}{endpoints} a partir de un modelo,
y al tenerlo desarrollado prácticamente me dio la funcionalidad \gls{crud} de toda
la aplicación casi al instante.

Además, las pruebas realizadas hasta el momento con la otra librería me permitieron
integrar una librería de gestión de autentificación \textemdash véase \ref{sec:tech:jwt}
\textemdash y otra de gestión de usuarios \textemdash \textit{friendsofsymfony/user-bundle}
\textemdash.

No obstante, en ese momento uno de los interesados de la aplicación me dijo que no
podía usar una versión de Symfony que no fuera la estable, la cual por aquel
entonces era la versión 3.4. El problema residía en que API Platform usaba una
versión de Symfony superior, y que yo no encontraba la forma de hacer que
los dos \glslink{webframework}{frameworks} fueran compatibles en la versión que
el interesado pedía. Así que mientras encontraba una solución, continué con
el trabajo de formación que había realizado, ya que esto le daba al menos un
\glslink{backend}{backend} funcional al alumno que se había incorporado.

Finalmente encontré dónde surgía la incompatiblidad: cuando intentaba integrar
la versión 3.4 de Symfony con la versión estándar de API Platform, este último
\glslink{webframework}{framework} se quejaba porque la versión de Symfony no
estaba entre sus candidatas para poder funcionar correctamente. Por lo tanto,
indagando en las dependencias de la versión estándar, descubrí que ésta
dependía de otro paquete \textemdash véase \ref{sec:tech:apipack} \textemdash el cual
proveía de toda la funcionalidad necesaria, y aceptaba la versión 3.4 de Symfony
para poder funcionar con ella.

Así que con alivio, pude volver a utilizar API Platform, aunque tuve que
trabajar a marchas forzadas para instalar cuidadosamente, una por una, el resto
de librerías necesarias y asegurándome de que cada paso quedaba bien registrado
en el repositorio donde tenía el código fuente de la aplicación. Al fin y al
cabo, al instalar la versión no estándar de API Platform, había ciertas librerías
que no estaban incluidas \textemdash sobre todo aquellas relativas a las pruebas
o a proporcionar integraciones con otras librerías \textemdash, tenía que repasar
qué es lo que faltaba para incluirlo en la aplicación.

Lo próximo fue meterse de lleno en aprender más sobre API Platform: cómo realizar
los denominados como subrecursos \textemdash véase el párrafo siguiente \textemdash, documentar la API \textemdash ya que a pesar de
que en esencia valía lo que aprendí con mis primeras pruebas de generación de
la documentación, API Platform lo integraba de otra forma \textemdash, configurar
Behat \textemdash véase \ref{sec:tech:behat} \textemdash y solucionar problemas
varios relacionados con la implementación, las pruebas y cumplir con ciertos
requerimientos que ya me empezaba a plantear el otro alumno \textemdash como que
cuando un usuario iniciaba sesión, se mandase también su identificador de usuario
para poder guardarlo y usarlo en peticiones futuras \textemdash.

Sobre los subrecursos: \textit{subresources} en inglés, son aquellos que permiten
obtener unos resultados filtrados con respecto a dos o más entidades. Por ejemplo, si
en un dominio existen las entidades \textit{User} y \textit{Book}, un
subrecurso útil podría ser el de \url{https://example.org/users/21/books} para
obtener la lista de libros del usuario con el identificador 21.

\paragraph{Febrero}
\textit{Nota: se puede encontrar el informe completo con las tareas detalladas
en el anexo \ref{app:reports:february}}
En el modelo algunas de las relaciones se establecieron del tipo \textit{``uno a uno''},
y por algún motivo daban problemas a la hora de insertar datos en la base de datos,
sobre todo con las restricciones de referencia a claves extranjeras en las
operaciones de eliminación. Se invirtió mucho tiempo en investigar qué podría estar
causando este problema, mirando las sentencias \gls{sql} que se generaban e intentando
reproducir paso a paso el error que se estaba produciendo, ya que curiosamente cuando
aplicaba una relación \textit{``uno a uno''} sin el \glslink{webframework}{framework}
API Platform el error no se daba, pero con esa librería en el proyecto sí. Finalmente
tras mucho indagar no pude dar con el problema, y a pesar de haber expuesto
el problema en los canales de soporte pertinentes, nadie parecía encontrar explicación
a lo que ocurría. No obstante el problema se pudo esquivar debido a que después
de analizar el \glslink{modentrel}{modelo entidad-erlación} y debatirlo con los
interesados, se debatió que tenía más sentido sustituir dichas relaciones por unas
del tipo \textit{``uno a n''}, como era en el caso de los presupuestos de un
proyecto, por ejemplo. Aunque la cosa no acabó aquí ya que se dedicó más tiempo a
intentar descubrir qué era lo que causaba el problema, y cómo podría solucionarse,
a pesar de que ya no hiciera falta saberlo, pero no hubo manera. Por lo que finalmente
asumí que probablemente se debería a mi torpeza, y decidí no perder más el tiempo con
ello, a pesar de no quedarme satisfecho.

Continué implementando las pruebas restantes para el resto de entidades de la aplicación,
ya que al estar el segundo alumno desarrollando su parte, no quería entorpecer su avance
y por lo tanto me centraba en implementar las cosas que me iba pidiendo, además de los
requerimientos que tenía que cumplir en la aplicación. Pero siempre con la idea en mente
de completar todas las pruebas, y finalmente pude hacerlo dejando la aplicación en un
estado de \textit{``completamente probada''}. Obviamente estas pruebas evolucionarían
y se modificarían en el futuro, y se añadirían nuevas según desarrollase nuevas cosas,
pero al menos me había quitado esa \textit{espinita}.

Uno de los elementos que supuso un cambio significativo fue la inclusión del
sistema de notificaciones de correo. Se utilizó la librería \textit{swiftmailer/swiftmailer},
debido a que es una librería completísima que facilita el envío de emails utilizando
diferentes protocolos de transporte, soporta cifrado e inicio de sesión en
servidores de correo, permite enviar mensajes en \gls{html}\cite{swiftmailer_symfony} y demás
funcionalidades. También porque es la librería recomendada por el propio \glslink{webframework}{framework}
Symfony, y otras muchas librerías más como por ejemplo la librería de gestión de usuarios
\textit{friendsofsymfony/user-bundle}, la cual utiliza el ya mencionado \textit{Swiftmailer} como
su sistema de gestión de notificaciones de correo electrónico. El problema principal residía en que
a pesar de que API Platform no recomendaba la librería de gestión de usuarios ya mencionada, sí
proveía de un \textit{bridge}\cite{apip_fosuser} o puente que facilitaba la integración en el \glslink{webframework}{framework}
de creación de \gls{api}s. No especificaba cuáles eran los motivos, pero descubrí
uno de ellos al empezar con el desarrollo de las notificaciones de correo: las
funcionalidades de recuperación de contraseña, registro, envío de email de confirmación
y demás requerían de la visita en un recurso web. Por lo tanto, su forma de implementar
las notificaciones, y las generaciones de \gls{url}s con identificadores únicos para
verificar la identidad del usuario que realiza las acciones, estaban muy enfocadas a
proporcionar dichos servicios accediendo a diferentes recursos web. El problema
fundamental residía en que yo estaba desarrollando una \gls{api} \gls{rest}, la
cual debería ser totalmente independiente del \glslink{frontend}{frontend} con el
que se realizan las peticiones, y por lo tanto el hecho de tener que obligar a que
las capas de visualización tuvieran que visitar ciertos recursos web lo consideraba
incoherente y limitante. La solución era tener que realizar \textit{``\glslink{method_overriding}{overrides}''}
de muchísimas partes de la librería de gestión de usuarios, para intentar adaptarla a
mis necesidades. Así que siguiendo el consejo de API Platform, decidí crear un sistema
de usuarios personalizado \textemdash denominado \textit{``custom user provider''} en
el mundo de Symfony\textemdash, justamente porque el trabajo de tener que crear dicho
sistema iba a ser muy similar al de tener que adaptar una librería que no estaba diseñada
para trabajar con \gls{api}s.

Una vez se implementó el sistema de usuarios y se integró con \textit{Swiftmailer},
se pasó a desarrollar un sistema de invitaciones para restringir el registro en la
aplicación: sólo los gestores con privilegios podrían invitar, mediante el envío
automatizado de un correo electrónico desde la aplicación, a ciertos usuarios a ser
parte de la aplicación. Estos usuarios tendrían que validar su correo haciendo clic en
el enlace correspondiente en su correo electrónico, que después el \glslink{frontend}{frontend}
traduciría en una petición a la \gls{api}.

En ése momento, tras revisar el dominio, me di cuenta de que había cometido algunos
errores en el diseño del mismo. No eran errores graves, ya que se podían corregir
fácilmente sin tener que desechar nada de lo desarrollado hasta la fecha, pero requerían
algunos cambios significativos que sobre todo iban a suponer un trabajo considerable
en cuanto a modificar todas las pruebas afectadas. La causa de estos errores se debe a lo siguiente:
antes de realizar el primer \glslink{modentrel}{modelo entidad-relación},
se me presentó un modelo de gestión de la empresa que habían desarrollado y que querían
que sustituyera el modelo de gestión que venían usando hasta entonces. Desarrollé el
\glslink{modentrel}{modelo entidad-relación} equivalente, y lo presenté a los interesados.
No obstante, al validarlo con casos reales y actuales de la empresa, mezclamos conceptos
de la aplicación de gestión que se estaba usando actualmente con los nuevos conceptos del modelo,
y eso unido al \glslink{modentrel}{modelo entidad-relación} diseñado generó muchísima confusión.
A pesar de todo ello, el modelo parecía válido porque no encontramos limitaciones a la hora
de validar los casos reales. Obviamente, y como menciono al principio del párrafo, al revisar
el modelo de nuevo encontré errores que achaqué a la confusión a la hora de debatir este tema,
y mi falta de clarividencia a la hora de plasmar lo que resultó de dichos
debates, aunque en mi defensa he de decir que todos los involucrados en los susodichos debates
salimos algo confundidos de ellos. Aún así, estos errores no suponían cambios muy
significativos, así que adapté el \glslink{modentrel}{modelo entidad-relación} al modelo
final que se presenta en la figura \ref{fig:erdiag}, y presenté los cambios a los interesados,
esta vez sin tanta confusión y con más facilidad debido a las adaptaciones realizadas. Las
consecuencias de estos cambios fueron el tener que adaptar las pruebas de la aplicación y los
correspondientes datos de prueba, lo que supuso un trabajo considerable. Por si eso fuera poco,
el consumo de memoria de las pruebas se disparó, o más bien me fijé que era muy alto. Tras
mucho trastear e investigar por si había realizado algo fuera de lo común que estuviera
malgastando recursos, pregunté en ciertos canales de desarrolladores que me comentaron que
no tenía nada de que preocuparme ya que los consumos estaban dentro de lo normal. De hecho,
de ahí a un tiempo, en una actualización de la librería de pruebas, el consumo descendió muy
significativamente, lo que me dio la prueba definitiva de que realmente ése consumo alto de
recursos no se debía a algún fallo cometido por mi parte.

Tras tenerlo todo atado, continué desarrollando las validaciones de los diferentes campos
de las entidades, ya que hasta este punto únicamente había realizado unas validaciones
genéricas.

Finalmente la última funcionalidad que comencé a implementar en el mes de febrero fue la de la
confirmación del registro por parte de los usuarios, ya que el sistema de invitaciones
de registro funcionaba bien, pero el desarrollo de la confirmación se paró por el asunto
de la modificación del dominio. Esto lo realicé con \gls{dto}s.
Es decir: la validación en el \glslink{webframework}{framework} Symfony se realiza
sobre las entidades \textemdash véase la sección de validaciones \ref{sec:tech:valid}\textemdash,
y por lo tanto estos \textit{DTO}s no son más que pequeñas entidades que sirven para un
propósito específico, como puede ser el recoger el identificador y el correo electrónico
de la confirmación del usuario.

\paragraph{Marzo}
\textit{Nota: se puede encontrar el informe completo con las tareas detalladas
en el anexo \ref{app:reports:march}}

En marzo seguí con las modificaciones en la serialización de las entidades, tanto para
limitar la información que se mostraba a cada rol del usuario como para satisfacer las
necesidades de navegación e información que me pedía el desarrollador del \glslink{frontend}{frontend}.

Después vino una tremenda refactorización de las pruebas de la aplicación. Hasta entonces,
las pruebas consistían en, generalmente, esperar unos valores específicos después de
realizar ciertas peticiones a la base de datos que contenía los datos de prueba. No obstante,
esto suponía un trabajo considerable y repetitivo a la hora de desarrollar una nueva
funcionalidad junto con sus pruebas, o realizar modificaciones a las funcionalidades ya existentes. Por
lo que decidí transformar las pruebas a pruebas de validación con JSON Schema \textemdash véase la sección  \ref{tech:sec:jsonschema} \textemdash
en gran parte, lo que facilitaba mucho la mantenibilidad y reutilización de los esquemas en distintas pruebas.

Finalicé el desarrollo del sistema de verificación de usuarios con el que había comenzado
a trastear a finales de febrero, y aproveché para implementar la funcionalidad de cambio
de contraseña y de cambio de email, todo ello desarrollado y validado con los \gls{dto}s.
Junto con esto, ideé unas pequeñas restricciones
e indicadores que ayudaban a identificar el estado de la cuenta cuando se estaba llevando
a cabo un proceso de cambio de contraseña, recuperación de contraseña o cambio de dirección de correo electrónico: por ejemplo, si se comenzaba
con el proceso de recuperar la contraseña pero el usuario iniciaba sesión antes de completarlo,
se entendía que o bien no lo había solicitado o que ya se había acordado de ella, y por lo tanto
el proceso se daba por finalizado sin que el usuario tuviera que hacer nada más.

Después de limitar la información enviada por cada rol de usuario, terminé de implementar el sistema
de control de acceso, el cual hasta el momento estaba mínimamente desarrollado para probar la restricción de
acceso a ciertos recursos. Terminó siendo muy sencillo gracias a la herencia de roles de Symfony,
que facilita la gestión de los mismos, y que permite hilar muy fino en cuanto a los recursos o
serie de recursos que se quieren restringir mediante el uso de expresiones regulares. Por ejemplo, 
se puede restringir la zona \url{https://example.org/admin} con una expresión regular \textit{\^/admin} que haga que todas las
subrutas como por ejemplo \url{https://example.org/admin/check_status} estén restringidas a un rol.
Los recursos que debían estar disponibles para los tres roles, pero que debían devolver una información u otra,
se restringían programando la lógica de negocio, y aplicando los grupos de serialización
\textemdash véase sección \ref{sec:serializer} \textemdash que se habían desarrollado para las
entidades. Con todo esto, se desarrollaron también las pruebas correspondientes, comprobando que
absolutamente todas las restricciones funcionaban como se pretendía que lo hicieran.

Las últimas funcionalidades que se desarrollaron antes de finalizar con la fase de desarrollo
fueron las de abrir y cerrar proyectos \textemdash restringiendo la posibilidad de añadir o
eliminar tareas por parte de ningún usuario, incluyendo los gestores \textemdash y permitir a
los usuarios que marquen ciertas tareas como favoritas \textemdash para ayudarles a guardar tareas que se
repiten muchas veces y poder reutilizarlas sin tener que volver a introducir todos los datos\textemdash.

El resto del tiempo se dedicó a ajustar la aplicación, arreglar fallos e implementar las
últimas peticiones del otro alumno para flexibilizar el \glslink{backend}{backend}.


\subsection{Calidad}
Se ha alcanzado la calidad mínima especificada en sección de la gestión de
calidad \ref{sec:qua:manag}. En cambio, los elementos de la calidad añadida
no han sido desarrollados, ya que no se ha dispuesto de los recursos suficientes
como para dedicarles un tiempo de desarrollo que fuera a culminar en una
nueva funcionalidad completamente probada. Si no hubiera tenido los desvíos indicados en
la sección de seguimiento y control del tiempo \textemdash véase sección 
\ref{sec:syc:time} \textemdash, creo que la única funcionalidad que pudiera
haber implementado es la onceava.

Finalmente no he podido utilizar la metodología \gls{tdd} en el desarrollo, ya que
generalmente mi intención era poder \textit{publicar} las funcionalidades cuanto
antes para que el otro alumno las tuviera listas para trabajar sobre ellas. Antes de
subir los cambios al repositorio que contenía el código de la aplicación, a veces solía
dejar el código de los cambios en un servidor de pruebas que usábamos para juntar los
dos desarrollos, y de mientras iba realizando las pruebas oportunas para que una vez
se hubiera probado que todo funcionaba bien, subir los cambios al repositorio. Esta
forma de trabajar me permitía tener un colchón de ventaja sobre el otro alumno y por lo
tanto podía trabajar más relajadamente. Por contra, la desventaja
de esta forma de trabajo era que si el otro alumno se ponía a desarrollar sobre una
funcionalidad que no estaba probada, podía ocurrir que después tuviera que cambiar
ligeramente su implementación para adaptarla a las correcciones aplicadas después
de realizar las pruebas, algo que sucedió en alguna ocasión.

\subsection{Riesgos}
Debido a que los riesgos identificados \ref{section:risks} no se han mostrado
significativamente a lo largo del desarrollo del proyecto, no se han tenido que
poner en marcha los planes de contingenciPor contra, la desventaja
de esta forma de trabajo era que si el otro alumno a \ref{section:continplans}.

Es cierto que hubo un momento en el que quizás podría considerarse que el riesgo
de 
\textit{``No tener un producto mínimo que sea funcional cuando el otro
integrante comience a desarrollar su parte.''} estuvo cerca de cumplirse, ya
que tal y como relato en la sección de seguimiento y control de tiempo, la
que corresponde al mes de enero \ref{sec:tnc:jan}, la semana en la que se
incorporó el otro alumno estaba todavía trasteando con las tecnologías de
formación y las del desarrollo del producto final. Aunque también he de decir
que el otro alumno siempre tuvo un \glslink{backend}{backend} operativo con
el que comenzar su desarrollo.

\section{Conclusiones}
Con respecto al producto, se ha logrado el objetivo de construir una \gls{api}
sobre un \glslink{backend}{backend} Symfony que permita ser consumida por
diferentes \glslink{frontend}{frontends}. La aplicación en sí podría resumirse
como un conjunto de librerías cohesionadas con las que se ha desarrollado
un código que realiza la tarea propuesta. Una de las funcionalidades que
quisiera destacar es que la aplicación en sí ha sido construida enteramente con
proyectos de \glslink{opensource}{código abierto}, y es por ello que me apena el no haber conseguido
convencer a los interesados para que el código de la aplicación terminara haciéndose
público bajo una licencia de software libre o código abierto, aunque entiendo perfectamente
los intereses corporativos.

La gestión del proyecto, en mi opinión, ha sido aceptable. Creo que tomé una buena
decisión en no invertir demasiado tiempo en intentar crear un plan de ruta, ya que
después de todo el proceso de desarrollo, no creo que me hubiera acercado a las
estimaciones iniciales. Lo que sí que es cierto es que se retrasó demasiado el
comienzo del desarrollo, y también la formación en la tecnología, lo que en caso
contrario podría haberme permitido desarrollar alguna funcionalidad más. El objetivo
de realizar un seguimiento y control exhaustivo se ha cumplido, ya que he tomado nota
de todo lo que he realizado a lo largo del desarrollo, aunque creo que para una
próxima vez sería mejor categorizar las tareas realizadas, ya que de esta forma
se podría elaborar un informe un poco más abstracto, que diera una mejor idea de
todo lo realizado sin tener que leer una por una todas la tareas que se llevaron a cabo.

En cuanto a los objetivos personales, estoy muy satisfecho de haber podido
trabajar en un proyecto con el \gls{webframework}{framework} Symfony, ya que uno
de mis objetivos era poder profundizar y entenderlo mucho mejor, y lo he conseguido.
También quería seguir las mejores prácticas posibles, y es por ello que entre otras
cosas he dedicado mucho tiempo a realizar las pruebas de las funcionalidades
implementadas, y a intentar seguir minuciosamente los patrones y estilos de código
recomendados, para que si en un futuro se abre el código fuente de la aplicación al público
no sólo no se pueda poner en evidencia a la empresa por no seguir las prácticas recomendadas,
sino que tampoco se me pueda poner en evidencia a mi por realizar un desarrollo chapucero. Sé
que esto último es muy subjetivo, y que a pesar de mis buenas intenciones puede que
no haya logrado mi objetivo, pero me quedo muy
tranquilo al saber que he hecho todo lo posible por crear un producto de calidad y
siguiendo lo que muchos expertos recomiendan y publican en las documentaciones
oficiales, blogs y salas de soporte de las diferentes tecnologías.

En general, creo que tanto los interesados como yo hemos terminado muy satisfechos
con el trabajo que este proyecto ha supuesto.

\section{Líneas de trabajo futuras}
Las líneas de trabajo futuras creo que van encaminadas a completar el desarrollo
de las funcionalidades opcionales que quedaron sin desarrollar, además de las
funcionalidades futuras que quieran incluir en la aplicación.

Cuando se me preguntó sobre cómo haría yo para comercializar un producto
como este, planteé lo siguiente: al ser una aplicación que no es en sí novedosa,
y de la que se pueden encontrar alternativas en el mercado, una idea podría ser
la de abrir el código fuente y convertir el proyecto en uno de
\glslink{opensource}{código abierto}, ofreciendo a su vez un servicio en el que
la empresa se haría cargo de mantener una instancia del producto por un determinado
precio, como por ejemplo hace GitLab\footnote{https://gitlab.com/}. Las ventajas
de este modelo de negocio es que cualquiera puede atreverse a auditar, mejorar,
sugerir correcciones, aportar críticas e incluso tomar parte en el desarrollo
del producto, además de que daría una buena imagen a la empresa por tener un proyecto
\glslink{opensource}{open source}. Las desventajas de este tipo de modelo de negocio
son que no se puede dejar un producto como este sin actualizar mucho tiempo, y sin
incorporar nuevas características o mejoras constantes, ya que un proyecto que tiene
pinta de no haberse actualizado desde hace tiempo resulta poco atractivo. Además, otra
de las desventajas es que hay que mimar el producto: para que no se ponga en duda
la profesionalidad de la empresa, sería recomendable seguir las mejores prácticas posibles y
documentar el código extensamente y de forma inteligible, entre otras cosas. Con todo esto,
se podría apostar por pelear con el producto en un mercado que ya está saturado de este
tipo de productos.

El servicio prestado tendría que ser fiable y escalable, de forma que los potenciales
clientes tuvieran la sensación de que merece la pena pagar por él. De forma que un
cliente pudiera gestionar varias empresas, podría hacerse para que cada vez que el
cliente diera de alta una de ellas, se lanzase una instancia de \gls{docker} con
el producto, e insertando automáticamente ciertos datos que son necesarios para empezar
a organizar la aplicación \textemdash básicamente la cuenta de un gestor con privilegios
\textemdash.

Por lo que, en resumen, estas propuestas esconden en sí más proyectos y líneas de trabajo
futuras sobre las que extender el ecosistema del producto desarrollado en este proyecto.
