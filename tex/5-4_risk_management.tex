\section{Gestión de riesgos}
\subsection{Planificación de la gestión de riesgos}
Los riesgos se identificarán y recogerán en una lista, y posteriormente
se elaborarán planes de contingencia para lidiar con los problemas que
los riesgos puedan plantear en un futuro.

\subsection{Identificación de riesgos}
\label{section:risks}
\begin{itemize}
    \item No cumplir con la calidad mínima especificada.
\end{itemize}
A pesar de que se tiene noción sobre las tecnologías que se verán
involucradas en el producto, nunca se ha realizado ningún desarrollo
similar en un entorno real con proyección a comercializar el producto.
Es por ello que puede resultar que las estimaciones realizadas sean
insuficientes y que no se satisfaga el alcance definido. Este riesgo
por lo tanto se considera un riesgo alto.

\begin{itemize}
    \item No tener un producto mínimo que sea funcional cuando el otro
        integrante comience a desarrollar su parte.
\end{itemize}
Relacionado con el riesgo anterior, puede darse la situación en la que
el producto no sea mínimamente funcional cuando el otro integrante
comience su desarrollo. Este riesgo se clasifica como de nivel medio.

\subsection{Planes de contingencia}
\label{section:continplans}
\begin{itemize}
    \item No cumplir con la calidad mínima especificada.
\end{itemize}
A la hora de desarrollar, se priorizarán aquellas funcionalidades y
especificaciones que tengan que ver con los requerimientos, y si aún y
todo el tiempo para cumplir con la calidad mínima empieza a escasear,
se centrará el desarrollo en dichas funcionalidades.

Si el problema es debido a la falta de recursos, se valorará la
aceptación del producto desarrollado con los interesados del producto.
En caso afirmativo, se dará por concluido el mismo y se aceptará el
cierre de la parte de desarrollo del proyecto. En caso negativo, el
proyecto se cerrará y se marcará como inacabado.

\begin{itemize}
    \item No tener un producto mínimo que sea funcional cuando el otro
        integrante comience a desarrollar su parte.
\end{itemize}
En este caso se abandonará cualquier otro trabajo para centrarse en
proporcionar al otro integrante un producto funcional con el que pueda
probar su desarrollo.

\subsection{Seguimiento y control de la calidad}
El seguimiento y control de los riesgos se realizará a lo largo del
proyecto, y se recogerán las incidencias, si hubiere, en un apartado
que indique la aplicación de los planes de contingencia propuestos y la
forma en la que se han llevado a cabo.
