\section{Gestión del alcance del proyecto}
La propuesta del proyecto incluye unas indicaciones generales de cómo quiere
la empresa que sea la aplicación, pero son insuficientes para poder realizar
un análisis realista del alcance que va a tener el proyecto.

Además, hay que tener en cuenta que el objetivo de este proyecto es realizar
una aplicación que funcione de forma cohesionada entre los diferentes
departamentos de la empresa, por lo que es muy probable que las
requerimientos cambien a lo largo del desarrollo del proyecto, o se tengan
que modificar.

\subsection{Planificación de la gestión del alcance}
El alcance del proyecto se generará una vez se conozcan, validen y contrasten
todos los requerimientos recogidos. Después, se generará la \gls{edt} una vez
se disponga de la primera versión del alcance, e incluirá todos los aspectos
del \gls{pfm}: planificación, gestión, desarrollo, seguimiento y control y
cierre.

El seguimiento y control del alcance se realizará a lo largo del desarrollo
del proyecto, revisando también los requerimientos, para evaluar si será
necesaria la aplicación de una modificación al alcance y al \gls{edt}. Antes de
hacer efectiva una modificación se tendrá que evaluar y comprobar que la
aplicación de dicha modificación no afecte a la viabilidad del proyecto.
Seguidamente se comprobará el \gls{edt} y se verificará si es necesario también
un cambio en dicha estructura.

\subsection{Planificación de la gestión de requerimientos}
\label{subsec:pgr}
Los trabajadores usan actualmente una aplicación determinada de seguimiento y
control de tiempos, clientes y proyectos. Es por ello que es crucial realizar
entrevistas con los trabajadores de los diferentes departamentos para extraer
de sus respuestas aquellas cosas que les gustan, las que no, y las que
mejorarían, además de su experiencia en general usando dicha aplicación. Al
procesar estas respuestas se podrán obtener los requerimientos del producto
a desarrollar y, finalmente, su validación consistirá en exponer dichos
requerimientos ante los interesados del proyecto, dando su visto bueno o no
a aquellos puntos con los que estén de acuerdo.

El seguimiento y control de los requerimientos se realizará a lo largo del
desarrollo del proyecto, cuando los diferentes interesados de la aplicación
contrasten que los requerimientos indicados y las que finalmente han sido
implementadas coinciden.

En caso de que los requerimientos tengan que sufrir algún cambio, se
evaluará si el nuevo requerimiento, o la modificación de uno ya existente,
es viable y se puede llevar a cabo. Se podrá modificar la prioridad e
importancia de los diferentes requerimientos, e incluso descartar unos
requerimientos por otros. En caso de que los cambios en los requerimientos
supongan un cambio drástico en el proyecto, se realizará una reunión con
los interesados del proyecto para ratificar dicho cambio.

\subsection{Recopilación de requerimientos}
\label{subsec:rdr}
Tras realizar el proceso descrito en \ref{subsec:pgr}, se validaron los
siguientes requerimientos que deberá tener la aplicación, ordenados por 
importancia:

\begin{enumerate}
\label{enum:req:lis}
 \item Desarrollar un \gls{backend} en forma de \gls{api} \gls{rest} que
 permita intercambiar datos con un \gls{frontend}

 \item El \gls{backend} ha de proporcionar control de acceso para restringir el
 contenido visible \textemdash ya sea restringiendo el acceso a recursos
 completos o limitando la información mostrada en recursos específicos
 \textemdash dependiendo de una serie de roles.
 
 \item La aplicación ha de dar la opción de gestionar clientes, y asignarles
 proyectos y grupos de proyectos a dichos clientes.
 
 \item La aplicación ha de permitir poder gestionar, a parte de los propios 
 proyectos, los presupuestos, bolsas de horas, usuarios, tareas y categorías 
 asociadas.
 
 \item La aplicación ha de desarrollarse de forma que pueda ser fácilmente 
 mantenible y que permita la extensión y mejora en el futuro.

 \item Los proyectos de la aplicación deberán tener una estructura común, y
 los gestores deberán tener la opción de modificar dicha estructura según
 las necesidades específicas de cada proyecto.

 \item La aplicación permitirá crear informes que recopilen datos por proyecto,
 departamento y por trabajador.

 \item La aplicación permitirá crear listas de favoritos, plantillas, o
 reordenar las tareas por usuario para facilitar la imputación de horas.

 \item La aplicación permitirá cerrar y reabrir los proyectos.

 \item La aplicación permitirá a los gestores de proyectos imputar horas a
 terceras personas, por ejemplo en el caso de que haya un trabajador por cuenta
 propia contratado.
 
 \item La aplicación permitirá enviar recordatorios a aquellos usuarios que
 se les haya olvidado imputar horas.

 \item La aplicación tendrá una integración con \gls{jira}

 \item La aplicación permitirá crear procesos a los gestores para facilitar
 la gestión.

 \item La aplicación permitirá visualizar de forma sencilla qué personas
 están de vacaciones.

 \item Todos los proyectos tendrán que compartir los mismos colores en las
 categorías comunes.

 \item Una aplicación de escritorio que facilite la imputación de horas
 y, que entre las funcionalidades estándar de la aplicación, tenga una que
 ayude a realizar el seguimiento de la tarea que se está realizando en
 ese momento mediante un botón \textit{play / stop}.

 \item La interfaz ha de ser rápida y sencilla de usar para imputar horas de
 tareas individuales y un grupo de tareas.

 \item La interfaz ha de mostrar a cada usuario cuántas horas ha imputado
 a lo largo de la semana y cuántas le restan por imputar para completar sus
 horas semanales.

 \item La interfaz ha de mostrar sugerencias para las imputaciones: que analice
 las últimas imputaciones para recomendar y ayudar a imputar cuando se trabaja
 en las mismas tareas o en los mismos proyectos.

 \item La interfaz ha de permitir a cada usuario limitar su vista de horas a la
 jornada laboral que realiza, ahorrando así el tener que deslizarse hasta la
 hora de inicio y hora de finalización de la jornada.

 \item La interfaz ha de permitir a cada usuario configurar la vista de las
 imputaciones: diaria, semanal y mensual.
\end{enumerate}

\subsection{Alcance}
El alcance de este proyecto abarca no sólo el desarrollo del producto en sí,
sino también la gestión del proyecto que dé como resultado dicho producto. Esto
incluye la planificación, diseño y desarrollo del producto, extracción de
conclusiones y líneas futuras, realización del seguimiento y control del
proyecto y la redacción de la documentación del proyecto.

En cuanto al producto en sí, las secciones de calidad mínima \ref{sec:min:qual}
y calidad añadida \ref{sec:add:qual}, recogen qué requisitos serán tenidos en
cuenta a la hora de desarrollar el producto. No obstante, es necesario explicar
que los requerimientos recogen especificaciones para el desarrollo tanto de
un \glslink{backend}{backend}, como de un \glslink{frontend}{frontend}, y que
este proyecto se centra únicamente en el desarrollo del
\glslink{backend}{backend}. Esto se debe a que en un principio se propuso
realizar el proyecto al completo, pero después de recoger los requerimientos
y especificar las calidades que debería de tener el producto, consideré que
el realizar el desarrollo de las dos partes culminaría en un resultado que
haría honor al refrán \textit{``quien mucho abarca, poco aprieta''}. Tras
exponer mis dudas a los interesados, se me propuso dividir la propuesta en
dos proyectos individuales, de los cuales uno de ellos se me asignó a mí, y
el otro proyecto se asignó a otro alumno de la misma facultad donde cursé
el máster.

\subsection{EDT}
\begin{verbatim}
Clepsydra
├ Gestión
│ └ Planificación
│   ├ Alcance
│   │ ├ Gestión del alcance
│   │ ├ Gestión de requerimientos
│   │ ├ Requerimientos
│   │ ├ Alcance
│   │ ├ EDT
│   │ ├ Gestión de la validación del alcance
│   │ └ Gestión del seguimiento y control del alcance
│   ├ Tiempo
│   │ ├ Gestión del tiempo
│   │ ├ Definición de actividades
│   │ ├ Secuencia de actividades
│   │ ├ Estimación de la duración de las actividades
│   │ └ Gestión del seguimiento y control de las actividades
│   ├ Calidad
│   │ ├ Gestión de la calidad
│   │ ├ Calidad mínima y calidad añadida
│   │ └ Gestión del seguimiento y control de la calidad
│   ├ Riesgos
│   │ ├ Gestión de riesgos
│   │ ├ Riesgos
│   │ └ Planes de contingencia
│   ├ Seguimiento y control
│   │ ├ Seguimiento y control del alcance
│   │ ├ Seguimiento y control del tiempo
│   │ ├ Seguimiento y control de la calidad
│   │ └ Seguimiento y control de los riesgos
│   └ Cierre
│     ├ Gestión del cierre
│     ├ Presentación del PFM
│     └ Cierre
├ Diseño
│ ├ Diseño de la base de datos
│ ├ Diseño de los casos de uso
│ └ Diseño y documentación de la API
└ Desarrollo
  ├ Formación
  ├ Implementación
  ├ Corrección de errores
  └ Testing
\end{verbatim}

\subsection{Validación del alcance}
La validación del alcance se hará junto con los requerimientos del proyecto,
en una reunión con los interesados del proyecto. En el caso de que se propongan
alteraciones se tomarán en consideración y se realizarán las modificaciones
oportunas acorde con lo expuesto.

En caso de que sea necesario modificar el alcance a lo largo del ciclo de
vida del proyecto, se tendrán que validar dichos cambios con los diferentes
interesados.

\subsection{Seguimiento y control del alcance}
\label{subsec:syc:scope}
El seguimiento del alcance se realizará periódicamente, cada Lunes de cada
semana, desde la fecha de inicio del desarrollo de la aplicación, comprobando
qué requerimientos se han cumplido, cuales faltan por cumplir, y cuales hay
que descartar en el caso de que no se dispongan de los suficientes recursos
para dedicar a dichos requerimientos.
