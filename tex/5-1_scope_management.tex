\section{Gestión del alcance del proyecto}
La propuesta del proyecto incluye unas indicaciones generales de cómo quiere
la empresa que sea la aplicación, pero son insuficientes para poder realizar
un análisis realista del alcance que va a tener el proyecto.

Además, hay que tener en cuenta que el objetivo de este proyecto es realizar
una aplicación que funcione de forma cohesionada entre los diferentes
departamentos de la empresa, por lo que es muy probable que las
requerimientos cambien a lo largo del desarrollo del proyecto, o se tengan
que modificar.

\subsection{Planificación de la gestión del alcance}
El alcance del proyecto se generará una vez se conozcan, validen y contrasten
todos los requerimientos recogidos. Después, se generará la \gls{edt} una vez
se disponga de la primera versión del alcance, e incluirá todos los aspectos
del \gls{pfm}: planificación, gestión, desarrollo, seguimiento y control y
cierre.

El seguimiento y control del alcance se realizará a lo largo del desarrollo
del proyecto, revisando también los requerimientos, para evaluar si será
necesaria la aplicación de una modificación al alcance y al \gls{edt}. Antes de
hacer efectiva una modificación se tendrá que evaluar y comprobar que la
aplicación de dicha modificación no afecte a la viabilidad del proyecto.
Seguidamente se comprobará el \gls{edt} y se verificará si es necesario también
un cambio en dicha estructura.

\subsection{Planificación de la gestión de requerimientos}
Los trabajadores usan actualmente una aplicación determinada de seguimiento y
control de tiempos, clientes y proyectos. Es por ello que es crucial realizar
entrevistas con los trabajadores de los diferentes departamentos para extraer
de sus respuestas aquellas cosas que les gustan, las que no, y las que
mejorarían, además de su experiencia en general usando dicha aplicación. Al
procesar estas respuestas se podrán obtener los requerimientos del producto
a desarrollar y, finalmente, su validación consistirá en exponer dichos
requerimientos ante los interesados del proyecto, dando su visto bueno o no
a aquellos puntos con los que estén de acuerdo.

El seguimiento y control de los requerimientos se realizará a lo largo del
desarrollo del proyecto, cuando los diferentes interesados de la aplicación
contrasten que los requerimientos indicados y las que finalmente han sido
implementadas coinciden.

En caso de que los requerimientos tengan que sufrir algún cambio, se
evaluará si el nuevo requerimiento, o la modificación de uno ya existente,
es viable y se puede llevar a cabo. Se podrá modificar la prioridad e
importancia de los diferentes requerimientos, e incluso descartar unos
requerimientos por otros. En caso de que los cambios en los requerimientos
supongan un cambio drástico en el proyecto, se realizará una reunión con
los interesados del proyecto para ratificar dicho cambio.

\subsection{Recopilación de requerimientos}

\subsection{Alcance}

\subsection{EDT}

\subsection{Validación del alcance}
El seguimiento del alcance se realizará periódicamente, cada Lunes de cada
semana, desde la fecha de inicio del desarrollo de la aplicación, comprobando
qué requerimientos se han cumplido, cuales faltan por cumplir, y cuales hay
que descartar en el caso de que no se dispongan de los suficientes recursos
para dedicar a dichas requerimientos.

\subsection{Seguimiento y control del alcance}
La verificación del alcance se hará junto con los interesados del proyecto:
se enumerarán todas y cada una de los requerimientos comprendidas en el alcance
y se comprobará la correcta implementación y funcionamiento de las mismas.
