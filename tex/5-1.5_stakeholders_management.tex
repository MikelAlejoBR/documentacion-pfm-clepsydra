\section{Gestión de los interesados}
\subsection{Planificación de la gestión de los interesados}
Se realizará una lista con los diferentes interesados en el proyecto, qué
rol tienen y cuánto peso tienen en el proyecto. Esta lista se podrá actualizar
si se identifica algún interesado más en pla aplicación a lo largo del
proyecto.

\subsection{Interesados en el proyecto}
Los interesados que he identificado son los siguientes:

\begin{itemize}
    \item Los gestores de la empresa.
    \item Los gestores de proyectos.
    \item Los usuarios de la aplicación.
    \item El alumno que se incorporará para desarrollar el \glslink{frontend}{frontend}.
    \item Yo mismo.
\end{itemize}

\subsubsection{Los gestores de la empresa}
Este grupo de interesados son los que han propuesto realizar este proyecto, y el
interés que tienen en él es que salga adelante con el objetivo de que les ayude a gestionar
mejor la empresa. Debido a que las decisiones que toman afectan al resto de interesados,
el peso de su opinión es alto.

\subsubsection{Los gestores de proyectos}
Este grupo de interesados depende de los proyectos que les asignen los gestores
de la empresa, y están encargados de gestionar los proyectos en sí: planificarlos,
y asignar usuarios y recursos a los mismos. El peso de su opinión se considera
medio-alto, ya que la facilidad a la hora de gestionar un proyecto
puede afectar significativamente el rendimiento del desarrollo del mismo.

\subsubsection{Los usuarios de la aplicación}
Los usuarios de la aplicación o los trabajadores, es el grupo que va a usar esta aplicación más intensivamente y
de forma más restrictiva, ya que serán los que se encarguen únicamente de apuntar
todas las tareas realizadas en los proyectos que estén asignados. Por el motivo
anteriormente expuesto, el peso de sus opiniones es bajo a la hora de tomar
decisiones en la aplicación.

\subsubsection{El alumno que se incorporará para desarrollar el frontend}
Este alumno, debido a que va a realizar el \glslink{frontend}{frontend} de la aplicación,
contará con un peso medio en cuanto a la toma de decisiones, ya que probablemente
terminará guiándome en cuanto a los cambios necesarios que requerirá la aplicación
para que la capa de presentación sea completa, fácil de usar y utilizable.

\subsubsection{Yo mismo}
Mi interés en este proyecto es muy alto, ya que de él depende que termine
mi máster, y también que la empresa termine contando con un producto de
calidad que pueda utilizar, modificar y extender en el futuro. Creo que mi
opinión en la toma de decisiones tiene un peso alto, ya que al ser el que
se va a enfrentar al desarrollo del producto, voy a ser el que actúe como
intermediario entre los deseos, ideas y expectativas del resto de los interesados
y la realidad del desarrollo, incluyendo sus virtudes y defectos.
