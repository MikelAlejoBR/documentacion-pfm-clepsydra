\newglossaryentry{accidentalComplexity}{
  name=Complejidad Accidental,
  description={
    Se llama así a la complejidad que no está asociada con la búsqueda y
    desarrollo de la solución en sí, sino a la que está asociada al uso de
    herramientas o el desarrollo de las mismas para alcanzar dicha solución
  }
}

\newglossaryentry{apig}{
  name=API,
  description={
    Grupo de procedimientos definidos que facilitan la comunicación entre
    componentes de software
  }
}

\newglossaryentry{backports}{
  name=Backports,
  description={
    En Debian, los \textit{backports} se denominan a los paquetes recompilados 
    de las ramas \textit{testing} y \textit{unstable} para poder ser instalados
    en las distribuciones Debian de la rama estable \cite{backports_debianwiki}
  }
}

\newglossaryentry{bddg}{
  name=Behavior-Driven Development,
  description={
    Se considera una extensión de \gls{tdd} que hace uso de un
    \glslink{lde}{lenguaje de dominio específico} para convertir sentencias
    escritas en lenguaje natural en pruebas ejecutables \cite{bdd_wiki}
  }
}

\newglossaryentry{backend}{
  name=Back End,
  description={
    Capa de lógica y acceso a datos del software
  }
}

\newglossaryentry{cookie}{
  name=Cookie,
  description={
    Pequeño fragmento de datos que las aplicaciones web envían a los usuarios,
    y que generalmente estos últimos almacenan en su navegador, con la finalidad
    de guardar cierta información de estado \textemdash como por ejemplo
    sesiones, opciones de personalización de la aplicación o la lista del
    carro de la compra digital\textemdash. Después esta información se manda
    de vuelta al servidor para que este último adecúe la aplicación al usuario
    que mandó la petición \cite{cookies_wiki}
  }
}

\newglossaryentry{dbalg}{
  name=DBAL,
  description={
    \gls{api} que tiene como objetivo unificar la comunicación entre la
    aplicación y las distintas tecnologías de bases de datos, como pueden ser
    MariaDB, MySQL, SQL Server, etc. La idea es proveer al desarrollador una
    única interfaz que le evite tener que programar una interfaz por
    tecnología de base de datos \cite{dbal_wiki}
  }
}

\newglossaryentry{dtog}{
  name=DTO,
  description={
    Un objeto que transmite datos entre procesos\cite{dto_wiki}
  }
}

\newglossaryentry{docker}{
  name=Docker,
  description={
    Docker es un software que realiza virtualización a nivel de sistema operativo,
    permitiendo lanzar \textit{``contenedores''} independientes desde una única
    instancia de Linux, lo que ahorra el coste de gestionar máquinas virtuales\cite{docker_wiki}
  }
}

\newglossaryentry{endpoint}{
  name=Endpoint,
  description={
    La \gls{url} de una \gls{api} o un \glslink{backend}{backend} que responden
    a una petición\cite{endpoint_so}
  }
}

\newglossaryentry{fork}{
  name=Fork,
  description={
    Se denomina \textit{fork} a la acción de clonar el código fuente de un 
    proyecto de software y empezar un desarrollo independiente del mismo, 
    creando un software distinto y separado del proyecto original 
    \cite{fork_wiki}
  }
}

\newglossaryentry{frontend}{
  name=Front End,
  description={
    Capa de visualización del software
  }
}

\newglossaryentry{hateoasg}{
  name=HATEOAS,
  description={
    Restricción de la arquitectura \gls{rest}, la cual pretende proveer de
    indicaciones al consumidor de la \gls{api} sobre qué pasos se pueden dar en
    ese punto mediante enlaces a las siguientes operaciones disponibles. En un
    contexto de una aplicación de un banco, la respuesta que un cliente
    recibiría al hacer una llamada a una operación de consulta de saldo, podría
    incluir los enlaces a las operaciones de \textit{transferencia},
    \textit{extracción de dinero} o \textit{consultar movimientos}, por ejemplo
    \cite{hateoas_so} \cite{hateoas_wiki}
  }
}

\newglossaryentry{method_overriding}{
  name=Method overriding,
  description={
    Es una característica que permite, en la programación orientada a objetos,
    proveer de una implementación específica de un método en una subclase, que
    sustituya la implementación del método de la súper clase \cite{method_over_wiki}
  }
}

\newglossaryentry{modentrel}{
  name=Modelo entidad-relación,
  description={
    Un modelo entidad-relación o diagrama \\ entidad-relación es una herramienta
    para el modelado de datos que permite representar las entidades relevantes
    de un sistema de información así como sus interrelaciones y propiedades
  }
}

\newglossaryentry{jira}{
  name=Jira,
  description={
    Software de seguimiento de proyectos e incidencias
  }
}

\newglossaryentry{kvmg}{
  name=Kernel Virtual Machine,
  description={
    Solución de virtualización para Linux en la que el kernel se convierte en 
    el hipervisor \textemdash software, firmware o hardware que se encarga 
    de gestionar y lanzar máquinas virtuales \cite{hypervisor_wiki} 
    \textemdash. Permite virtualizar múltiples sistemas operativos Linux, 
    Windows, BSD, OS X y más, mediante una \gls{api} con la que los susodichos 
    sistemas operativos pueden interactuar. \cite{kvm_wiki}
  }
}


\newglossaryentry{lde}{
  name=Lenguaje de dominio específico,
  description={
    El \textit{LDE} o más conocido como \textit{DSL} \\ \textemdash Domain
    Specific Language\textemdash por sus siglas en inglés, es un lenguaje
    informático especializado en un dominio específico, con el objetivo de
    facilitar la representación de problemas y la resolución de los mismos
    \cite{dsl_wiki}
  }
}

\newglossaryentry{ormg}{
  name=ORM,
  description={
    Esta técnica permite realizar mapeos de objetos y sus atributos a tablas
    relacionales
  }
}

\newglossaryentry{opensource}{
  name=Software Open-Source,
  description={
    Tipo de software del cual su código fuente es públicamente accesible,
    y que tiene una licencia de copyright asociada que permite estudiar,
    cambiar el código fuente y distribuirlo a cualquier persona y por cualquier
    motivo. Generalmente este software se suele desarrollar de una manera
    colaborativa\cite{opensource_wiki}
  }
}

\newglossaryentry{pruebascajanegra}{
  name=Pruebas de caja negra,
  description={
    Las pruebas de caja negra son aquellas que analizan la funcionalidad de una
    aplicación sin tener en cuenta la estructura interna o el funcionamiento de
    dicha funcionalidad \cite{blackbox_wiki}, centrándose en proporcionar a la
    funcionalidad una entrada y comprobando que la salida sea la esperada
  }
}

\newglossaryentry{restg}{
  name=REST,
  description={
    Los servicios web \gls{rest} son una herramienta que facilitan el
    intercambio de información, y el uso de dicha información intercambiada
    entre sistemas informáticos en internet. Dichos servicios hacen uso de una
    serie de operaciones sin estados, que devuelven un resultado al consumidor
    de dicho servicio, generalmente en formato XML, HTML, JSON o cualquier otro
    formato definido
  }
}

\newglossaryentry{semanticweb}{
  name=Web semántica,
  description={
    La web semántica es una extensión de la web que mediante estándares
    propuestos por la W3C, pretenden impulsar formatos de datos comunes, como
    pueden ser el RDFa, JSON-LD o HAL. De esta forma, se intenta impulsar una
    web contextualizada en la que los datos tengan relación entre sí, incluso
    si dichos datos están diseminados por distintos recursos web \cite{semweb}
  }
}

\newglossaryentry{swaggerui}{
  name=Swagger UI,
  description={
    Interfaz de usuario que muestra de forma muy amigable las operaciones de
    una \gls{api}, además de ofrecer la posibilidad de interactuar con dichas
    operaciones en la misma página donde se muestran. Esta documentación se
    genera a partir de una especificación denominada Swagger
  }
}

\newglossaryentry{testsfuncionales}{
  name=Pruebas funcionales,
  description={
    Las pruebas funcionales o \textit{functional testing} se consideran como
    un proceso para asegurar la calidad del software, y también como un
    tipo de pruebas de la modalidad \glslink{pruebascajanegra}{caja negra}.
    Generalmente, este tipo de pruebas se realizan introduciendo una entrada y
    analizando la salida de la funcionalidad en cuestión que se está probando,
    comprobando que todo funciona según las especificaciones de dicha
    funcionalidad \cite{functionalt_wiki}
  }
}

\newglossaryentry{webframework}{
  name=Web Framework,
  description={
    Un \textit{Web Framework} es una abstracción en la que un software que
    provee de una funcionalidad genérica puede ser selectivamente modificada
    adiciones, modificaciones o eliminaciones de código. Estos
    \textit{frameworks} ayudan en el desarrollo de aplicaciones web, incluyendo
    servicios web, recursos web y \gls{api}s web. El objetivo de estas
    herramientas es el de automatizar ciertas tareas y ahorrar tiempo en las
    actividades comunes asociadas al desarrollo web
  }
}

\newglossaryentry{tddg}{
  name=Test Driven Development,
  description={
    El \gls{tdd} es un proceso de desarrollo de software que se basa en la
    repetición de un ciclo de desarrollo muy corto: los requerimientos se
    convierten en casos de pruebas muy específicos, y únicamente está permitido
    programar para hacer que dichos casos de prueba se realicen
    satisfactoriamente
  }
}
