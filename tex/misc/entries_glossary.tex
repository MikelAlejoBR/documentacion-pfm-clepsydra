\newglossaryentry{apig}{
  name=API,
  description={
    Grupo de procedimientos definidos que facilitan la comunicación entre
    componentes de software
  }
}

\newglossaryentry{backend}{
  name=Back End,
  description={
    Capa de lógica y acceso a datos del software
  }
}

\newglossaryentry{frontend}{
  name=Front End,
  description={
    Capa de visualización del software
  }
}

\newglossaryentry{modentrel}{
  name=\small Modelo entidad-relación,
  description={
    Un modelo entidad-relación o diagrama entidad-relación es una herramienta
    para el modelado de datos que permite representar las entidades relevantes
    de un sistema de información así como sus interrelaciones y propiedades
  }
}

\newglossaryentry{jira}{
  name=Jira,
  description={
    Software de seguimiento de proyectos e incidencias
  }
}

\newglossaryentry{restg}{
  name=REST,
  description={
    Los servicios web \gls{rest} son una herramienta que facilitan el
    intercambio de información, y el uso de dicha información intercambiada
    entre sistemas informáticos en internet. Dichos servicios hacen uso de una
    serie de operaciones sin estados, que devuelven un resultado al consumidor
    de dicho servicio, generalmente en formato XML, HTML, JSON o cualquier otro
    formato definido
  }
}

\newglossaryentry{webframework}{
  name=Web Framework,
  description={
    Un \textit{Web Framework} es una abstracción en la que un software que
    provee de una funcionalidad genérica puede ser selectivamente modificada 
    adiciones, modificaciones o eliminaciones de código. Estos
    \textit{framewokrs} ayudan en el desarrollo de aplicaciones web, incluyendo
    servicios web, recursos web y \gls{api}s web. El objetivo de estas
    herramientas es el de automatizar ciertas tareas y ahorrar tiempo en las
    actividades comunes asociadas al desarrollo web
  }
}

\newglossaryentry{tddg}{
  name=Test Driven Development,
  description={
    El \gls{tdd} es un proceso de desarrollo de software que se basa en la
    repetición de un ciclo de desarrollo muy corto: los requerimientos se
    convierten en casos de pruebas muy específicos, y únicamente está permitido
    programar para hacer que dichos casos de prueba se realicen
    satisfactoriamente
  }
}
